\documentclass[11pt,a4paper]{report}

\begin{document}
\title{Distributed Chat Application: Final Report}
\author{Monarchs of Coding:\\
  Aigerim Serikbekova\\
  Aydin Akyol\\
  Fareed Qaseem\\
  Spyridon Kousidis\\
  Ozhan Azizi\\
  Vijendra Patel}
\date{March 2017}
\maketitle


\chapter{Introduction}

Chat systems refers to communication over the Internet, communication that offers a real-time transmission of messages from sender to receiver(s). There are many types of messages such as texts, multi-media, etc. In this project, we have been developing a reliable, secure, real-time, distributed chat system where users can exchange messages among each other. Our application can be accessed using PCs and Android mobile phones.
The system is made of three sub-systems mainly, which are: A server written in Elixir, a Java Desktop application, and an Android application. With a friendly graphical user interface.
At the beginning of the work, we defined several features that our system should deliver, some of them were priorities so the system can deliver a basic communication, the others were non-priorities which we could achieve some and left some for future work.

\section{Priority Features}
	\begin{itemize}
		\item Registering New Profile.
		\item Sign-in Process.
		\item Contacts Look-up.
  		\item Exchanging Text Messages.
 		\item Listing Existing Chats.
		\item Chats History.
		\item Sign-out Process.
	\end{itemize}

\section{Non-priority Features}
	\begin{itemize}
		\item Deleting Existing Chats.
		\item Managing a User's Profile.
		\item Participation in Group Chats.
		\item Exchangin Multimedia Messages.
	\end{itemize}

The whole developing process was made using Agile development methodology, we had several steps to develop every feature, starting from \emph{Backlog} which contains the ideas and sub ideas, \emph {Specify} which contains the notes and criteria we should satisfy during next steps, \emph{Implement} which is the programming phase, \emph{Test/QA} where the implemented features are tested against the criteria which was set in \emph{Specify} step, and finally \emph{Deployed} which means that this feature is completed.



\chapter{Review of Current Market}

% Can talk about WhatsApp, FB Messenger, WeChat, Kik etc.

\section{Messaging protocols}

\subsection{XMPP}

\subsection{MQTT}

\section{Encrypted communication protocols}

\subsection{Signal}


\chapter{Requirements and Design}

% Does overall system design go in here? Check final report handout.

\chapter{Implementation}

\subsection{Toolkit}

\begin{itemize}
  \item Docker:
  \item pyinvoke: A task runner in Python.
\end{itemize}

\section{Android Client}

\subsection{Libraries}

\section{Desktop Client}

\subsection{Libraries}

\section{Backend}

The backend for this application has been created with the Elixir programming language. Elixir is a functional language that uses the Erlang VM(Virtual Machine) to run its code \cite{website:elixir_homepage}. The primary reason choosing Elixir was because of its built-in distributed capabilities, which allows every erlang node to be aware of any publish-subscribe channel we created and thus the backend can easily scale horizontally without some nodes losing topic information. Without this functionality, we could have created a Redis cluster to handle the publish-subscribe channels, however this would have made the backend more complex.

\subsection{Libraries}

\begin{tabular}{ | l | p{11cm} |}
  \hline Package name & Usage \\ \hline
  phoenix & todo \\
  phoenix\_pubsub & todo \\
  phoenix\_ecto & todo \\ \hline
  postgrex & todo \\
  comeonin & todo \\
  guardian & todo \\
  jose & todo \\ \hline
  excoveralls & Coverage reporting \\
  credo & Static analysis \\
  dogma & Static analysis \\
  \hline
\end{tabular}

\subsubsection{Phoenix Framework}


\subsubsection{Static Analysis}

Credo and Dogma

\subsection{Amazon Web Services with Hashicorp's Terraform}


\chapter{Team Work}

\chapter{Evaluation}

% What we've achieved.

\section{Security}

% To what extent this system is secure.

\chapter{Peer Review}

\bibliographystyle{plain}
\bibliography{final-report}{}
\end{document}
